\documentclass[a4paper]{article}
\usepackage{times}
\usepackage{mathptmx}
\usepackage[xtable,dvipsnames]{xcolor}
\usepackage[boldfont,slantfont]{xeCJK}
\setmainfont{Times New Roman}
\setsansfont{Times New Roman}
\setCJKmainfont{宋体}
\setCJKsansfont{黑体}
\setCJKmonofont{仿宋}
\setCJKfamilyfont{hei}{黑体}
\setCJKfamilyfont{kai}{楷体}
\setCJKfamilyfont{xw}{华文新魏}
\setCJKfamilyfont{cu}{方正粗宋简体}
\newcommand{\hei}{\CJKfamily{hei}}
\newcommand{\kai}{\CJKfamily{kai}}
\newcommand{\xw}{\CJKfamily{xw}}
\newcommand{\cu}{\CJKfamily{cu}}
\usepackage{amsmath,amssymb,amsthm}
\usepackage{geometry}
\geometry{left=2.5cm,right=2.5cm,top=2.5cm,bottom=2.5cm}
\usepackage{listings}
\lstset{language=C++,breaklines=true,extendedchars=false,numbers=left,basicstyle=\ttfamily,numberstyle=\ttfamily,tabsize=4,showstringspaces=false,escapechar=`,frame=shadowbox,keywordstyle=\color{BrickRed},rulesepcolor=\color{red!20!green!20!blue!20}}
\usepackage{titlesec}
\titleformat{\section}{\Large \hei}{\thesection}{1em}{}
\titleformat{\subsection}{\large \hei}{\thesubsection}{1em}{}
\usepackage[colorlinks,linkcolor=black]{hyperref}
\usepackage{pgf,tikz}
\usetikzlibrary{calc,intersections}
\usetikzlibrary{arrows,snakes,backgrounds}
\usetikzlibrary{shapes.geometric}
\pgfmathdeclarefunction{fix}{1}{\global \pgfmathunitsdeclaredfalse}
\usepackage{indentfirst}
\def\tcdots{\,\!\cdots\!\,}
\def\leq{\leqslant}
\def\geq{\geqslant}
\def\byh{\!\!“}
\def\eyh{”\!\!}
\def\bkh{\!\!(}
\def\ekh{)\!\!}
\def\C{\mathrm{C}}
\def\A{\mathrm{A}}
\title{算法竞赛中的数学问题——讲义}
\author{东北育才学校\quad 张听海}
\def\dou{,\!\!}
\begin{document}
    \renewcommand{\baselinestretch}{1.25}\normalsize
    \setlength{\parindent}{2em}
    \setlength{\abovedisplayskip}{1pt}
    \setlength{\belowdisplayskip}{1pt}
    \pagestyle{plain}

    \maketitle

    \tableofcontents

    \newpage

    \section{快速幂}

    {\hei 理论基础:\!\!}
    
    (1)若$x\equiv y\pmod{p}$\dou 则$x^n\equiv y^n\pmod{p}$;\!\!

    (2)二进制与十进制的相互转化、位运算.

    {\hei 代码:\!\!}求$x^n\bmod p$.

    \begin{lstlisting}
int ans=1;
while(n){
    if(n&1) ans=ans*x%p;
    n>>=1;
    x=x*x%p;
}
    \end{lstlisting}

    {\hei 思想:\!\!}我们把正整数$n$写成二进制的形式\dou 如$n=11={(1011)}_2$\dou 则我们可以把$x^{11}$拆成$x^8\cdot x^2\cdot x^1$来计算.

    {\hei 具体实现:\!\!}我们用变量a保存中间结果$x^{2^i}$\dou 如果n\&1就乘入ans\dou 否则不乘入ans\dou 同时指数n通过{\ttfamily n>>=1}来更新\dou 中间结果a通过{\ttfamily a=a*a}来进行更新.

    {\hei 时间复杂度:\!\!}$O(\log n)$.

    {\hei 特别提醒:\!\!}这是一个绝大多数OI选手都熟练掌握而且写得都差不多的算法. 如果在考试时使用\dou 建议使用一些具有个人特色的变量名\dou 防止被认定为{\hei 代码雷同}.

    \section{素数筛法}

    \subsection{朴素的素数判断}

    {\hei 代码:\!\!}

    \begin{lstlisting}
for(int i=2; i<=n; i++){
    bool flag=1;
    for(int j=2; j<=sqrt(i+0.5); j++)
        if(i%j==0) flag=0;
    ...
}
    \end{lstlisting}

    {\hei 时间复杂度:\!\!}$O\big(n\sqrt{n}\big)$\dou 复杂得不得了.

    {\hei 原因:\!\!}对于某个正整数$i$\dou 它被小于等于$\sqrt{i}$的每个正整数都试除了一遍——这是完全没有必要的.

    {\hei 改进:\!\!}对于每个正整数$i$\dou 尝试只用它的质因子去试除.

    \subsection{Eratosthenes筛法}

    {\hei 代码:\!\!}

    \begin{lstlisting}
int m=sqrt(n+0.5);
memset(vst,0,sizeof(vst));
for(int i=2; i<=m; i++)
    if(!vst[i])    `//`!vst[i]`表示`i`为素数`.
        for(int j=i*i; j<=n; j+=i) vst[j]=1;
    \end{lstlisting}

    {\hei 算法的思想:\!\!}引入一个vst数组\dou 每找到一个素数$i$\dou 就筛掉大于等于$i^2$的所有$i$的倍数(为什么不用考虑$i^2$之前$i$的倍数?)\dou 这样筛去合数的过程变得具有目的性.

    {\hei 时间复杂度:\!\!}$O(n\log\log n)$\dou 已经比较理想了.

    {\hei 继续优化的可能:\!\!}事实上\dou 对于每个正整数$i$\dou 它被它的所有质因子都筛掉了一遍——我们可以尝试让每个正整数$i$都只被它的某一个正整数筛掉.

    \subsection{线性筛法}

    {\hei 代码:\!\!}

    \begin{lstlisting}
bool check[MAXN]={0};
int prime[MAXN]={0};
int tot=0;
for(int i=2; i<=N; i++){
    if(!check[i]) prime[tot++]=i;
    for(int j=0; j<tot&&i*prime[j]<=N; j++){
        check[i*prime[j]]=1;
        if(i%prime[j]==0) break;    `//关键在于理解此句`.
    }
}
    \end{lstlisting}

    {\hei 对于代码的解释:\!\!}这里我们用prime数组来记录已经找到的所有素数\dou check数组记录每个正整数是否为素数(记为0). 顺次扫描$2\sim N$中的每个正整数\dou 如果扫到某个正整数$i$的时候满足!check[i]\dou 则$i$为素数\dou 加入素数队列prime. 同时\dou 无论$i$是否为素数\dou 都筛掉$ip_0,ip_1,\tcdots$\dou 直到$p_j\mid i$. 这样的话\dou 我们就可以保证{\hei 每个正整数都只被它最小的素数筛掉}\dou 从而一定程度地优化了算法.

    {\hei 时间复杂度:\!\!}$O(n)$.

    {\hei 如何理解$p_j\mid i$:\!\!}如果$p_j\mid i_0$\dou 那么对于$p_j$后的任意$p_k$\dou $p_j\mid i_0p_k$\dou 我们不妨让$i_0p_k$在$i=\dfrac{i_0p_k}{p_j}(>i_0)$时被$ip_j$筛掉. 这样就避免了重复筛掉同一个正整数. 事实上\dou 这样做之后\dou 每个正整数$i$都会被它的最小质因子筛掉.

    {\hei 拓展——求欧拉函数$\varphi(n)$的值:\!\!}

    (1)欧拉函数$\varphi(n)$:\!\!对于任意正整数$n$\dou 我们用$\varphi(n)$表示$1\sim n$中与$n$互质的正整数的个数. 欧拉函数是一个非常重要的数论函数.

    (2)性质:\!\!

    $\bullet$对于素数$p$\dou $\varphi(p)=p-1$;\!\!

    $\bullet$对于互质的正整数$x,y$\dou $\varphi(xy)=\varphi(x)\varphi(y)$;\!\!

    {\color{white}$\bullet$} 特别地\dou 若$x=p$为质数\dou 且$p\nmid y$\dou 则$\varphi(py)=\varphi(p)\varphi(y)=(p-1)y$;\!\!

    $\bullet$对于素数$p$和与它不互质的正整数$x$\dou $\varphi(px)=p\varphi(x)$;\!\!

    $\bullet$对于任意正整数$n=p_1^{\alpha_1}p_2^{\alpha_2}\cdots p_n^{\alpha_n}$\dou $\varphi(n)=n\bigg(1-\dfrac{1}{p_1}\bigg)\!\bigg(1-\dfrac{1}{p_2}\bigg)\cdots\bigg(1-\dfrac{1}{p_n}\bigg)$.

    (3)代码:\!\!只需对线性筛法的代码稍作改动.

    \begin{lstlisting}
bool check[MAXN]={0};
int prime[MAXN]={0};
int tot=0;
int phi[MAXN]={0};
phi[1]=1;    `//不要忘记`.
for(int i=2; i<=N; i++){
    if(!check[i]){
        prime[tot++]=i;
        phi[i]=i-1;
    }        
    for(int j=0; j<tot&&i*prime[j]<=N; j++){
        check[i*prime[j]]=1;
        if(i%prime[j]==0){
            phi[i*prime[j]]=phi[i]*prime[j];
            break;
        }
        else phi[i*prime[j]]=phi[i]*(prime[j]-1);
    }
}
    \end{lstlisting}

    (4)时间复杂度:\!\!同线性筛法\dou $O(n)$.

    \section{费马小定理与欧拉定理}

    \subsection{费马小定理}

    {\hei 叙述:\!\!}已知正整数$a$和质数$p$\dou 且$\gcd (a,p)=1$\dou 则$a^{p-1}\equiv 1\pmod{p}$.

    {\hei 应用:\!\!}结合快速幂求乘法逆元\dou 即$a$模$p$的乘法逆元为$a^{p-2}$.

    \subsection{欧拉定理}

    {\hei 叙述:\!\!}若正整数$a,n$互质\dou 则$a^{\varphi(n)}\equiv 1\pmod{n}$.

    {\hei 广义欧拉定理:\!\!}对于任何整数$x$和$n\geq\varphi(k)$\dou 则$x^n\equiv x^{n\%\varphi(k)+\varphi(k)}\pmod{k}$.

    \section{欧几里得除法}
    
    \subsection{普通的欧几里得除法}

    {\hei 问题:\!\!}已知正整数$a,b$\dou 求$a,b$的最大公约数.

    {\hei 理论基础:\!\!}$\gcd(a,b)=\gcd(b,a\%b)$\dou 我们可以通过有限次这样的操作求出$\gcd(a,b)$.

    {\hei 代码:\!\!}

    \begin{lstlisting}
int gcd(int a,int b){
    if(b==0) return a;
    else return gcd(b,a%b);
}
    \end{lstlisting}

    \subsection{扩展欧几里得除法}

    {\hei 问题:\!\!}已知正整数$a,b$和它们的最大公约数$d$\dou 求方程$ax+by=d$的一组整数解$(x,y)$.

    {\hei 理论基础:\!\!}{\kai \bkh 裴蜀定理\ekh}若正整数$a,b$满足$\gcd(a,b)=d$\dou 则必存在整数$x,y$满足$ax+by=d$.

    特别地\dou 正整数$a,b$互质的充要条件为方程$ax+by=1$有整数解$(x,y)$.

    {\hei 代码:\!\!}

    \begin{lstlisting}
void gcd(int a,int b,int& d,int& x,int& y){
    if(b==0) {d=a,x=1,y=0;}
    else{
        gcd(b,a%b,d,y,x);
        y-=x*(a/b);    `//关键在于理解此句`.
    }
}
    \end{lstlisting}

    {\hei 关于{\ttfamily y-=x*(a/b)}的解释:\!\!}已知正整数$a,b$满足$\gcd(a,b)=d$\dou 且整数$x_0,y_0$满足$ax_0+by_0=d$.

    设$a/b=p,a\% b=r$\dou 这里\byh $/$\eyh 按计算机中的整除理解. 则有$(bp+r)x_0+by_0=d$\dou 整理得$$rx_0+b(px_0+y_0)=d.$$

    当从下一层递归中回到上一层时\dou 传回的参数为$x=x_0,y=px_0+y_0$. 而在上一层递归中我们想得到$x=x_0,y=y_0$\dou 因此需要{\ttfamily y-=x*(a/b)}.

    {\hei 应用:\!\!求乘法逆元}

    (1)乘法逆元:\!\!若$ab\equiv 1\pmod{p}$\dou 则$a,b$互为模$p$的乘法逆元.

    (2)在扩展欧几里得算法中取$b=p$\dou 得到的$x$即为$a$模$p$的乘法逆元.

    \section{三分法}

    {\hei 问题:\!\!}求单峰函数的最值.

    {\hei 思想:\!\!}以求先增后减的单峰函数$f(x)$在区间$[l,r]$上的极值点\dou 取$x_1,x_2$把区间分成等长的三个子区间$[l,x_1],[x_1,x_2],[x_2,r]$. 若$f(x_1)>f(x_2)$\dou 则极值点肯定不在区间$[x_2,r]$上\dou 进而问题变为求$f(x)$在区间$[l,x_2]$上的最值. 由此一步一步缩小范围\dou 最后得到极值点的近似值.

    {\hei 代码:\!\!}

    \begin{lstlisting}
double f(double x);
double solve(double l, double r){
    double x1=l+(r-l)/3;
    double x2=r-(r-l)/3;
    if(fabs(x1-x2)<0.0001) return f(x1);
    if(f(x1)>f(x2)) return solve(l,x2);
    else return solve(x1,r);
}
    \end{lstlisting}

    \section{排列组合}

    {\hei 数学基础:\!\!}
    
    (1)排列数$\A_m^n$:\!\!从$m$个互不相同的物品中选出$n$个排成一列\dou 共有$\A_m^n$种方式.

    (2)组合数$\C_m^n$:\!\!从$m$个互不相同的物品中选出$n$个\dou 共有$\C_m^n$种方式.

    (3)公式:\!\!$\A_m^n=\dfrac{m!}{(m-n)!},\C_m^n=\dfrac{m!}{n!(m-n)!}$.

    (4)排列数与组合数的关系:\!\!$\A_m^n=\C_m^n\cdot\A_n^n$.

    (5)杨辉三角:\!\!${(a+b)}^n$的各项展开式系数.

    \begin{center}
        1

        1\quad 1

        1\quad 2\quad 1

        1\quad 3\quad 3\quad 1

        1\quad 4\quad 6\quad 4\quad 1
    \end{center}

    性质:\!\!从第三行起\dou 两侧为1\dou 中间的每个数等于它\byh 肩上\eyh 的两个数之和.

    \subsection{Catalan数}

    (1)基本性质:\!\!$h_0=1,h_1=1,h_n=h_0h_{n-1}+h_1h_{n-2}+\cdots+h_{n-1}h_0$.

    如果能把公式化成上面这种形式\dou 则为Catalan数.

    (2)通项:\!\!$h_n=\C_{2n}^n-\C_{2n}^{n+1}=\dfrac{1}{n+1}\C_{2n}^n$.

    (3)证明:\!\!折线法.

    设定如下情境:\!\!在平面直角坐标系中\dou 从$A(0,0)$走到$B(n,n)$\dou 每次可向右走1单位或向上走1单位. 但在任意时刻\dou 已向右走的步数不少于向上走的步数. 记满足这样条件的路径数为$h_n$.
    
    从递推角度考虑. 假设一条合法路径与$y=x$\bkh 图中实线\ekh 的最后一个交点\bkh $B$之前\ekh 为$N(k,k)$\dou 则$A-N$的合法路径数为$h_k$.

    \begin{center}
        \begin{tikzpicture}
            \coordinate [label=below left:$A$] (A) at (0,0);
            \coordinate [label=above right:$B$] (B) at (3,3);
            \draw [dashed] (A) [step=0.5] grid (B);
            \draw (A)--(B);
            \draw [thick] (0,0)--(0.5,0)--(0.5,0.5)--(1.5,0.5)--(1.5,1)--(2.5,1)--(2.5,2)--(3,2)--(3,3);
            \coordinate [label=above left:$N$] (N) at (0.5,0.5);
            \coordinate [label=below right:$C$] (C) at (1,0.5);
            \coordinate [label=right:$D$] (D) at (3,2.5);
            \draw (C)--(D);
            \foreach \point in {A,B,C,D,N}
                \fill (\point) circle (1pt);
        \end{tikzpicture}
    \end{center}

    从$N$点出发的一步必须向右走\dou 整个路径的最后一步必须向上走. 之间由于路径与$y=x$再无交点\dou 则$C(k+1,k)-D(n,n-1)$可看作是$n'=n-k-1$的一个子问题\dou 因此路径数为$h_{n-k-1}$. 所以此时共有$h_kh_{n-k-1}$条路径.
    
    $k$的可能取值有$0,1,\cdots,n-1$\dou 累加得总路径数$\displaystyle h_n=\sum_{k=0}^{n-1}h_kh_{n-k-1}$.

    \begin{center}
        \begin{tikzpicture}
            \coordinate [label=below left:$A$] (A) at (0,0);
            \coordinate [label=above right:$B$] (B) at (3,3);
            \draw [dashed] (0,0) [step=0.5] grid (3,3.5);
            \draw (A)--(B);
            \draw [dashed] (0,0.5)--(3,3.5);
            \draw [thick] (0,0)--(0.5,0)--(0.5,1)--(1.5,1)--(1.5,2)--(3,2)--(3,3);
            \coordinate [label=above left:$M$] (M) at (1.5,2);
            \coordinate [label=above right:$B'$] (B') at (2.5,3.5);
            \draw [thick,BrickRed] (1.5,2)--(1.5,3.5)--(2.5,3.5);
            \foreach \point in {A,B,B',M}
                \fill (\point) circle (1pt);
        \end{tikzpicture}
    \end{center}

    另一方面\dou 显然我们只能在方格表在$y=x$右下方的部分运动. 不合法的路径将与$y=x+1$\bkh 图中虚线\ekh 相交.

    对于每条不合法的路径\dou 设其与$y=x+1$的最后一个交点为$M$\dou 将$M-B$部分的路线沿$y=x+1$翻折\dou 得到一条到达$B'(n-1,n+1)$的路径.

    显然\dou $A-B$的不合法路径与$A-B'$的路径一一对应.

    $A-B$的路径\bkh 含不合法的\ekh 共有$\C_{2n}^n$条\dou 而$A-B'$的路径共有$\C_{2n}^{n+1}$条\dou 得$h_n=\C_{2n}^n-\C_{2n}^{n+1}=\dfrac{1}{n+1}\C_{2n}^n$.

    (4)应用:\!\!

    $\bullet$出栈次序:\!\!一个栈\dou 现在要求将$1,2,\cdots,n$依次入栈\dou 求有多少种不同的出栈顺序.

    同证明中的模型对应\dou 入栈即为向右走1单位\dou 出栈即为向上走1单位\dou 栈中元素不少于0个即为任意时刻向右走的步数不少于向上走的步数.

    $\bullet$二叉树构成问题:\!\!一个$n$个顶点的二叉树\bkh 节点无编号\dou 但左右叶子不可颠倒\ekh 共有多少种?

    考虑根节点的左右子树\dou 若根的左子树有$k$个节点\dou 则右子树有$n-k-1$个节点\dou $k=0,1,\cdots,n-1$\dou 显然为卡特兰数模型.

    \subsection{Lucas定理}

    {\hei 目的:\!\!}大组合数取模.

    {\hei 基本形式:\!\!}$\C_n^m\equiv \C_{n/p}^{m/p}\C_{n\% p}^{m\% p}\pmod{p}$\dou $p$必须为素数.

    {\hei 证明:\!\!}注意到对任意$0<f<p$\dou 有$p\C_{p-1}^{f-1}=f\C_p^f$\dou 而$p$为素数\dou 故$p\mid\C_p^f$.

    设$n=sp+t,m=qp+r$\dou 其中$0<t,r<p$.

    则由二项式定理$${(1+x)}^n={\big[{(1+x)}^p\big]}^s\cdot{(1+x)}^t\equiv {\big(1+x^p\big)}^s\cdot{(1+x)}^t\pmod{p},$$注意到上式中\dou \byh $\equiv$\eyh 左右两个多项式只相差一些系数被$p$整除的项\dou 因此各项系数模$p$同余.

    比较$m$次项系数\dou 得$\C_n^m\equiv\C_s^q\C_t^r\pmod{p}$\dou 即$\C_n^m\equiv \C_{n/p}^{m/p}\C_{n\% p}^{m\% p}\pmod{p}$.

    {\hei 递归:\!\!}{\ttfamily Lucas(n,m)=C(n\%p,m\%p)*Lucas(n/p,m/p)\%p}.

    递归出口{\ttfamily if(m==0) return 1;}.

    {\hei 代码:\!\!}
    
    \begin{lstlisting}
int fac[N];
fac[1]=1;
for(int i=2; i<=N; i++) fac[i]=fac[i-1]*i%p;
long long pow(long long a,long long b){
    long long ans=1;
    while(b){
        if(b&1) ans=ans*a%p;
        b>>=1;
        a=a*a%p;
    }
    return ans;
}
long long C(long long n,long long m){
    if(m>n) return 0;
    return fac[n]*pow(fac[m]*fac[n-m],p-2)%p;
}
long long Lucas(long long n,long long m){
    if(m==0) return 1;
    return (C(n%p,m%p)*Lucas(n/p,m/p))%p;
}
    \end{lstlisting}

    {\hei 时间复杂度:\!\!}$O(p\log_pm)$.

    \section{矩阵乘法}

    {\hei 数学基础:\!\!}
    
    (1)矩阵:\!\!一个按照长方阵列排列的复数或实数集合.

    (2)矩阵乘法:\!\!设$A$为$m\times p$的矩阵\dou $B$为$p\times n$的矩阵\dou 那么称$m\times n$的矩阵$C$为矩阵$A$与$B$的乘积\dou 记作$C=AB$\dou 其中矩阵$C$中的第$i$行第$j$列元素可以表示为$${(AB)}_{ij}=\sum_{k=1}^pa_{ik}b_{kj}=a_{i1}b_{1j}+a_{i2}b_{2j}+\cdots+a_{ip}b_{pj}.$$

    例:\!\!设$A=
        \begin{bmatrix}
            1 & 2 & 3 \\
            4 & 5 & 6
        \end{bmatrix}
    $\dou $B=
        \begin{bmatrix}
            1 & 4 \\
            2 & 5 \\
            3 & 6
        \end{bmatrix}
    $. 则$$C=AB=
        \begin{bmatrix}
            1\times 1+2\times 2+3\times 3 & 1\times 4+2\times 5+3\times 6 \\
            4\times 1+5\times 2+6\times 3 & 4\times 4+5\times 5+6\times 6
        \end{bmatrix}
    =
        \begin{bmatrix}
            14 & 32 \\
            32 & 77
        \end{bmatrix}.$$

    (3)矩阵乘法的性质:\!\!

    $\bullet$乘法结合律:\!\!$(AB)C=A(BC)$;\!\!

    $\bullet$乘法左分配律:\!\!$(A+B)C=AC+BC$;\!\!

    $\bullet$乘法右分配律:\!\!$C(A+B)=CA+CB$;\!\!

    $\bullet$对数乘的结合性:\!\!$k(AB)=(kA)B=A(kB)$;\!\!

    $\bullet$矩阵乘法一般{\hei 不满足交换律}.

    {\hei 应用:\!\!}矩阵快速幂提高递推效率.

    例:\!\!求斐波那契数列第$n$项$(a_1=a_2=1,a_{n+2}=a_{n+1}+a_n)$.

    $$\begin{bmatrix}
        a_{n+1} & a_n
    \end{bmatrix}
    \begin{bmatrix}
        1 & 1 \\
        1 & 0
    \end{bmatrix}
    =
    \begin{bmatrix}
        a_{n+1}+a_n & a_{n+1}
    \end{bmatrix}
    =
    \begin{bmatrix}
        a_{n+2} & a_{n+1}
    \end{bmatrix}.$$

    可以先求出$n-2$个$
    \begin{bmatrix}
        1 & 1 \\
        1 & 0
    \end{bmatrix}$连乘的结果\dou 再与$
    \begin{bmatrix}
        1 & 1
    \end{bmatrix}$相乘即得到$
    \begin{bmatrix}
        a_n & a_{n-1}
    \end{bmatrix}$.

    可以把$O(n)$的递推降到$O(\log n)$.

    \section{高斯消元}

    {\hei 目的:\!\!}解多元一次方程组.

    {\hei 过程:\!\!}以解三元一次方程组$
    \left\{
        \begin{aligned}
            &2x+y+z=1,\\
            &6x+2y+z=-1,\\
            &\!-\!2x+2+z=7
        \end{aligned}
    \right.$为例.

    以上方程组可记为$
        \begin{bmatrix}
            x & y & z & val \\
            2 & 1 & 1 & 1 \\
            6 & 2 & 1 & -1 \\
            -2 & 2 & 1 & 7
        \end{bmatrix}$.

    用第一行消掉后两行的$x$项$
        \begin{bmatrix}
            x & y & z & val \\
            2 & 1 & 1 & 1 \\
            0 & -1 & -2 & -4 \\
            0 & 3 & 2 & 8
        \end{bmatrix}$.

    再用第二行消掉第三行的$y$项$
        \begin{bmatrix}
            x & y & z & val \\
            2 & 1 & 1 & 1 \\
            0 & -1 & -2 & -4 \\
            0 & 0 & -4 & -4
        \end{bmatrix}$.

    进而可从第三行得到$z=1$\dou 从下至上回代\dou 依次得到$y=2,x=-1$.

    {\hei 注意事项:\!\!}

    (1)系数不一定是整数——用double;\!\!

    (2)无解情况:\!\!某一行系数均为0\dou 但val不为0;\!\!

    (3)多解情况:\!\!存在某一行系数和val均为0.
\end{document}