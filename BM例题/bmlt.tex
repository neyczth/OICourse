\documentclass{beamer}
\usepackage{times}
\usepackage{mathptmx}
\usepackage[boldfont,slantfont]{xeCJK}
\setmainfont{Times New Roman}
\setsansfont{Times New Roman}
\setCJKmainfont{黑体}
\setCJKsansfont{黑体}
\setCJKmonofont{仿宋}
\setCJKfamilyfont{li}{隶书}
\setCJKfamilyfont{xk}{华文行楷}
\setCJKfamilyfont{xw}{华文新魏}
\newcommand{\li}{\CJKfamily{li}}
\newcommand{\xk}{\CJKfamily{xk}}
\newcommand{\xw}{\CJKfamily{xw}}
\usepackage{amsmath,amssymb,amsthm}
\definecolor{cmykgreen000}{cmyk}{0.5,0,1,0.2}
\usecolortheme[named=cmykgreen000]{structure}
\usetheme{Copenhagen}
\useoutertheme{miniframes}
\usefonttheme[onlymath]{serif}
\title{\xw 算法竞赛中的数学问题——例题}
\author{\li 东北育才学校\quad \xk 张听海}
\AtBeginSection[]{
    \begin{frame}[shrink]
        \tableofcontents[sectionstyle=show/shaded,subsectionstyle=show/shaded/hide]
    \end{frame}
}
\AtBeginSubsection[]{
    \begin{frame}[shrink]
        \tableofcontents[sectionstyle=show/shaded,subsectionstyle=show/shaded/hide]
    \end{frame}
}
\def\tcdots{\,\!\cdots\!\,}
\def\bkh{\!\!(}
\def\ekh{)\!\!}
\def\leq{\leqslant}
\def\geq{\geqslant}
\def\dou{,\!\!}
\hypersetup{pdfpagemode={FullScreen}}
\begin{document}
    \renewcommand{\baselinestretch}{1.25}\normalsize
    \setlength{\parindent}{0em}
    \setlength{\abovedisplayskip}{1pt}
    \setlength{\belowdisplayskip}{1pt}
    
    \maketitle

    \begin{frame}[shrink]
        \tableofcontents[hideothersubsections]
    \end{frame}

    \section{因数与倍数}

    \subsection{例1. [UOJ48]核聚变反应强度}

    \begin{frame}[fragile,shrink]
        \frametitle{例1. [UOJ48]核聚变反应强度}

        \begin{block}

            给出$n$个正整数$a_1,a_2,\tcdots,a_n$\dou 计算$a_1$与每个$a_i$的次大公约数\bkh 能同时整除$x,y$的正整数中第二大的数\ekh\dou 如果没有输出$-1$.\pause

            $n\leq {10}^5$\dou $a_i\leq {10}^{12}$.\pause
        \end{block}

        对于两个正整数$a_1$和$a_i$\dou 它们的公约数必为$\gcd(a_1,a_i)$的公约数. 即求$\gcd(a_1,a_i)$的次大公约数.\pause

        欧拉筛预处理出质数数列.\pause

        对于每个$a_i$\dou 欧几里得算法求出$\gcd(a_1,a_i)$\dou 之后从小到大用质数试除\dou 找到最小的$p\mid\gcd(a_1,a_i)$\dou 输出$\dfrac{\gcd(a_1,a_i)}{p}$.
    \end{frame}

    \subsection{例2. [POJ3696] The Luckiest Number}

    \begin{frame}[fragile,shrink]
        \frametitle{例2. [POJ3696] The Luckiest Number}
        \begin{block}

            对于给定的整数$L$\dou 找出$L$能整除最短的全8序列的长度.\pause

            注:\!\!全8序列为形如$\underbrace{888\cdots 8}_{n\text{个}8}$.\pause

            多组数据.\pause

            $1\leq L\leq 2\times {10}^9$.\pause
        \end{block}

        $\underbrace{888\cdots 8}_{n\text{个}8}=\dfrac{8}{9}\big({10}^n-1\big)=L\cdot p$\dou 即${10}^x-1=\dfrac{9Lp}{8}$.\pause

        设$m=\dfrac{9L}{\gcd(L,8)}$\dou 则存在$p'$使得${10}^x-1=mp'$\dou 即求${10}^x\equiv 1\pmod{m}$的最小解.\pause

        当$\gcd(10,m)\neq 1$时\dou 无解.
    \end{frame}

    \begin{frame}[fragile,shrink]
        \frametitle{例2. [POJ3696] The Luckiest Number}
        当$\gcd(10,m)=1$时\dou 由于${10}^{\varphi(m)}\equiv 1\pmod{m}$\dou 只需考虑$\varphi(m)$的因子.\pause

        对$\varphi(m)$质因数分解.\pause

        对每个质因子$p_i$\dou 执行$n=n/p_i$直到以下情形之一被满足:
        
        $(1)$ $p_i\nmid n$;\!\!$(2)$ $x^n\not\equiv 1\pmod{m}$.\pause

        考虑过全部质因子后即得解.
    \end{frame}

    \section{欧拉函数}

    \subsection{例3. [BZOJ2818]Gcd}

    \begin{frame}[fragile,shrink]
        \frametitle{例3. [BZOJ2818]Gcd}
        \begin{block}

            给定整数$N$\dou 求$1\leq x,y\leq N$且$\gcd(x,y)$为素数的数对$(x,y)$有多少对?\pause

            $1\leq N\leq {10}^7$.\pause
        \end{block}

        欧拉筛法预处理质数数列及$\varphi(n)$前缀和.\pause

        题目等价于求$1\leq x,y\leq\bigg\lfloor\dfrac{N}{p}\bigg\rfloor$且$\gcd(x,y)=1$的数对的个数\dou 其中$p$为质数.
    \end{frame}

    \subsection{例4. [BZOJ3884]上帝与集合的正确用法}

    \begin{frame}[fragile,shrink]
        \frametitle{例4. [BZOJ3884]上帝与集合的正确用法}
        \begin{block}

            求$2^{2^{2^{2^{\cdots}}}}\bmod p$的值.\pause

            $T$组数据.\pause

            $T\leq 1,000$\dou $1\leq p\leq {10}^7$.\pause
        \end{block}

        欧拉筛预处理欧拉函数值.\pause

        设$p=2^k\cdot q$\dou 其中$q$为奇数. 则$2^{2^{2^{2^{\cdots}}}}\bmod p=2^k\bigg(2^{2^{2^{2^{\cdots}}}-k}\bmod q\bigg)$.\pause

        由欧拉定理$2^k\bigg(2^{2^{2^{2^{\cdots}}}-k}\bmod q\bigg)=2^k\bigg[2^{\big(2^{2^{2^{\cdots}}}-k\big)\bmod\varphi(q)}\bmod q\bigg]$.\pause

        递归计算\dou 直至$q=1$.
    \end{frame}

    \subsection{例5. 离散对数问题}

    \begin{frame}[fragile,shrink]
        \frametitle{例5. 离散对数问题}
        \begin{block}

            已知$a,b,n$\dou 解同余方程$a^x\equiv b\pmod{n}$\dou 其中$\gcd(a,n)=1$.\pause
        \end{block}

        由欧拉定理\dou $a^x\equiv a^{x+\varphi(n)}\pmod{n}$. 因此只需枚举$0\leq x<\varphi(n)$.\pause
        
        分块优化. 设$x=p\big\lceil\varphi(n)\big\rceil-q$\dou 这里$0<p,q\leq\lceil\varphi(n)\big\rceil$.\pause

        则$a^{p\lceil\varphi(n)\rceil-q}\equiv b\pmod{n}$等价于$a^{p\lceil\varphi(n)\rceil}\equiv b\cdot a^q\pmod{n}$.\pause

        枚举$0<q\leq\lceil\varphi(n)\big\rceil$\dou 用hash表记录余数与$q$值的关系.\pause

        再枚举$p$\dou 找到使$x$最小的$p,q$.\pause

        以上算法也被称为大步小步法$($BSGS$)$.
    \end{frame}

    \section{三分法}

    \subsection{例6. [BZOJ1857]传送带}

    \begin{frame}[fragile,shrink]
        \frametitle{例6. [BZOJ1857]传送带}
        \begin{block}

            在二维平面上有两个线段型传送带$AB$和$CD$\dou 小明在传送带$AB$上的速度为$P$\dou 在传送带$CD$上的速度为$Q$\dou 在平面其余位置的速度为$R$\dou 求小明从$A$走到$B$需要的最短时间.\pause

            $1\leq P,Q,R\leq 10$\dou 各点坐标$\leq 1,000$.\pause
        \end{block}

        三分套三分.
    \end{frame}

    \section*{谢谢大家}

    \subsection*{谢谢大家}
    
    \begin{frame}[fragile]
        谢谢大家.
    \end{frame}
\end{document}