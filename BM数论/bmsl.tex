\documentclass[xcolor=dvipsnames]{beamer}
\usepackage{times}
\usepackage{mathptmx}
\usepackage[boldfont,slantfont]{xeCJK}
\setmainfont{Times New Roman}
\setsansfont{Times New Roman}
\setCJKmainfont{黑体}
\setCJKsansfont{黑体}
\setCJKmonofont{仿宋}
\setCJKfamilyfont{li}{隶书}
\setCJKfamilyfont{xk}{华文行楷}
\setCJKfamilyfont{xw}{华文新魏}
\newcommand{\li}{\CJKfamily{li}}
\newcommand{\xk}{\CJKfamily{xk}}
\newcommand{\xw}{\CJKfamily{xw}}
\usepackage{amsmath,amssymb,amsthm}
\usepackage{listings}
\lstset{language=C++,breaklines=true,extendedchars=false,numbers=left,basicstyle=\scriptsize\ttfamily,numberstyle=\scriptsize\ttfamily,tabsize=4,showstringspaces=false,escapechar=`,keywordstyle=\color{BrickRed},frame=shadowbox,rulesepcolor=\color{red!20!green!20!blue!20}}
\definecolor{cmykgreen000}{cmyk}{0.5,0,1,0.2}
\usecolortheme[named=cmykgreen000]{structure}
\usetheme{Copenhagen}
\useoutertheme{miniframes}
\usefonttheme[onlymath]{serif}
\title{\xw 算法竞赛中的数学问题——基础知识}
\author{\li 东北育才学校\quad \xk 张听海}
\AtBeginSection[]{
    \begin{frame}[shrink]
        \tableofcontents[sectionstyle=show/shaded,subsectionstyle=show/shaded/hide]
    \end{frame}
}
\AtBeginSubsection[]{
    \begin{frame}[shrink]
        \tableofcontents[sectionstyle=show/shaded,subsectionstyle=show/shaded/hide]
    \end{frame}
}
\def\leq{\leqslant}
\def\geq{\geqslant}
\def\byh{\!\!“}
\def\eyh{”\!\!}
\def\tcdots{\,\!\cdots\!\,}
\def\dou{,\!\!}
\hypersetup{pdfpagemode={FullScreen}}
\begin{document}
    \renewcommand{\baselinestretch}{1.25}\normalsize
    \setlength{\parindent}{0em}
    \setlength{\abovedisplayskip}{1pt}
    \setlength{\belowdisplayskip}{1pt}

    \maketitle

    \begin{frame}[shrink]
        \tableofcontents[hideothersubsections]
    \end{frame}

    \section{快速幂}

    \subsection{快速幂}

    \begin{frame}[fragile,shrink]
        \frametitle{快速幂}

        \begin{lstlisting}
int ans=1;
while(n){
    if(n&1) ans=ans*x%p;
    n>>=1;
    x=x*x%p;
}
        \end{lstlisting}\pause

        时间复杂度:\!\!$O(\log n)$.
    \end{frame}

    \section{素数筛法}

    \subsection{朴素的素数判断}

    \begin{frame}[fragile,shrink]
        \frametitle{朴素的素数判断}

        \begin{lstlisting}
for(int i=2; i<=n; i++){
    bool flag=1;
    for(int j=2; j<=sqrt(i+0.5); j++)
        if(i%j==0) flag=0;
    ...
}
        \end{lstlisting}\pause

        时间复杂度:\!\!$O\big(n\sqrt{n}\big)$.
    \end{frame}

    \subsection{Eratosthenes筛法}

    \begin{frame}[fragile,shrink]
        \frametitle{Eratosthenes筛法}

        \begin{lstlisting}
int m=sqrt(n+0.5);
memset(vst,0,sizeof(vst));
for(int i=2; i<=m; i++)
    if(!vst[i])    `//`!vst[i]`表示`i`为素数`.
        for(int j=i*i; j<=n; j+=i) vst[j]=1;
        \end{lstlisting}\pause

        特点:\!\!从质因子的角度考虑\dou 显然质因子的数目比正整数的数目少. 这样就大大降低了时间复杂度.\pause

        时间复杂度:\!\!$O(n\log\log n)$.
    \end{frame}

    \subsection{线性筛法}

    \begin{frame}[fragile,shrink]
        \frametitle{线性筛法}

        \begin{lstlisting}[basicstyle=\tiny\ttfamily,numberstyle=\tiny\ttfamily]
bool check[MAXN]={0};
int prime[MAXN]={0};
int tot=0;
for(int i=2; i<=N; i++){
    if(!check[i]) prime[tot++]=i;
    for(int j=0; j<tot&&i*prime[j]<=N; j++){
        check[i*prime[j]]=1;
        if(i%prime[j]==0) break;    `//关键在于理解此句`.
    }
}
        \end{lstlisting}\pause

        {\small 如果$p_j\mid i_0$\dou 那么对于$p_j$后的任意$p_k$\dou $p_j\mid i_0p_k$\dou 我们不妨让$i_0p_k$在$i=\dfrac{i_0p_k}{p_j}(>i_0)$时被$ip_j$筛掉. 这样就避免了重复筛掉同一个正整数. 事实上\dou 这样做之后\dou 每个正整数$i$都会被它的最小质因子筛掉.}
    \end{frame}

    \begin{frame}[fragile,shrink]
        \frametitle{线性筛法拓展——求欧拉函数}

        \begin{lstlisting}
bool check[MAXN]={0};
int prime[MAXN]={0};
int tot=0;
int phi[MAXN]={0};
phi[1]=1;    `//不要忘记`.
for(int i=2; i<=N; i++){
    if(!check[i]){
        prime[tot++]=i;
        phi[i]=i-1;
    }        
    for(int j=0; j<tot&&i*prime[j]<=N; j++){
        check[i*prime[j]]=1;
        if(i%prime[j]==0){
            phi[i*prime[j]]=phi[i]*prime[j];
            break;
        }
        else phi[i*prime[j]]=phi[i]*(prime[j]-1);
    }
}
        \end{lstlisting}
    \end{frame}

    \section{两个重要定理}

    \subsection{费马小定理}

    \begin{frame}[fragile,shrink]
        \frametitle{费马小定理}

        已知正整数$a$和质数$p$\dou 且$\gcd (a,p)=1$\dou 则$a^{p-1}\equiv 1\pmod{p}$.\pause

        应用:\!\!结合快速幂求乘法逆元\dou 即$a$模$p$的乘法逆元为$a^{p-2}$.
    \end{frame}

    \subsection{欧拉定理}

    \begin{frame}[fragile,shrink]
        \frametitle{欧拉定理}

        若正整数$a,n$互质\dou 则$a^{\varphi(n)}\equiv 1\pmod{n}$.\pause

        广义欧拉定理:\!\!对于任何整数$x$和$n\geq\varphi(k)$\dou 则$$x^n\equiv x^{n\%\varphi(k)+\varphi(k)}\pmod{k}.$$
    \end{frame}

    \section{欧几里得算法}

    \subsection{普通的欧几里得算法}

    \begin{frame}[fragile,shrink]
        \frametitle{普通的欧几里得算法}

        \begin{lstlisting}
int gcd(int a,int b){
    if(b==0) return a;
    else return gcd(b,a%b);
}
        \end{lstlisting}
    \end{frame}

    \subsection{拓展欧几里得算法}

    \begin{frame}[fragile,shrink]
        \frametitle{拓展欧几里得算法}
        
        裴蜀定理:\!\!若正整数$a,b$满足$\gcd(a,b)=d$\dou 则必存在整数$x,y$满足$ax+by=d$.\pause

        特别地\dou 正整数$a,b$互质的充要条件为方程$$ax+by=1$$有整数解$(x,y)$.
    \end{frame}

    \begin{frame}[fragile,shrink]
        \frametitle{拓展欧几里得算法}
        \begin{lstlisting}[basicstyle=\tiny\ttfamily,numberstyle=\tiny\ttfamily]
void gcd(int a,int b,int& d,int& x,int& y){
    if(b==0) {d=a,x=1,y=0;}
    else{
        gcd(b,a%b,d,y,x);
        y-=x*(a/b);    `//关键在于理解此句`.
    }
}
        \end{lstlisting}\pause
        
        {\small 已知正整数$a,b$满足$\gcd(a,b)=d$\dou 且整数$x_0,y_0$满足$ax_0+by_0=d$.\pause

        设$a/b=p,a\% b=r$\dou 这里\byh $/$\eyh 按计算机中的整除理解. 则有$(bp+r)x_0+by_0=d$\dou 整理得$rx_0+b(px_0+y_0)=d$.\pause

        当从下一层递归中回到上一层时\dou 传回的参数为$x=x_0,y=px_0+y_0$. 而在上一层递归中我们想得到$x=x_0,y=y_0$\dou 因此需要{\ttfamily y-=x*(a/b)}.}
    \end{frame}

    \begin{frame}[fragile,shrink]
        \frametitle{应用:\!\!求乘法逆元}

        $(1)$乘法逆元:\!\!若$ab\equiv 1\pmod{p}$\dou 则$a,b$互为模$p$的乘法逆元.\pause

        $(2)$在扩展欧几里得算法中取$b=p$\dou 得到的$x$即为$a$模$p$的乘法逆元.
    \end{frame}

    \section{三分法}

    \subsection{三分法}

    \begin{frame}[fragile,shrink]
        \frametitle{三分法}

        \begin{lstlisting}
double f(double x);
double solve(double l, double r){
    double x1=l+(r-l)/3;
    double x2=r-(r-l)/3;
    if(fabs(x1-x2)<0.0001) return f(x1);
    if(f(x1)>f(x2)) return solve(l,x2);
    else return solve(x1,r);
}
        \end{lstlisting}
    \end{frame}

    \section*{谢谢大家}

    \subsection*{谢谢大家}

    \begin{frame}[fragile]
        谢谢大家.
    \end{frame}
\end{document}